\chapter{Introduction to Algebra}

\section{Basic Definitions}
Algebra is the study of mathematical symbols and the rules for manipulating these symbols.

\begin{definition}
A \textbf{group} $(G, \cdot)$ is a set $G$ equipped with a binary operation $\cdot$ such that:
\begin{enumerate}
    \item Closure: For all $a, b \in G$, we have $a \cdot b \in G$.
    \item Associativity: For all $a, b, c \in G$, we have $(a \cdot b) \cdot c = a \cdot (b \cdot c)$.
    \item Identity: There exists an element $e \in G$ such that for all $a \in G$, we have $e \cdot a = a \cdot e = a$.
    \item Inverses: For each $a \in G$, there exists an element $a^{-1} \in G$ such that $a \cdot a^{-1} = a^{-1} \cdot a = e$.
\end{enumerate}
\end{definition}

\section{Examples of Groups}
\begin{itemize}
    \item $(\mathbb{Z}, +)$ is a group.
    \item $(\mathbb{Q}^*, \times)$ is a group.
\end{itemize}

\end{document}
